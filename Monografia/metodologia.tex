\chapter{Metodologia}
\label{c.metodologia}

\section{Métodos e Etapas}
\label{s.metodoseetapas}

Para o desenvolvimento do projeto foi realizado o levantamento bibliográfico de metodologias existentes para realizar Testes de Penetração, analisando seus diversos aspectos e quais nos auxiliariam melhor em nosso objetivo, bem como modos de como aplicá-las em Sistemas na Nuvem..

Com a base teórica definida a etapa seguinte foi aplicar uma estrutura modular que correspondesse ao modelo teórico, para que todas as etapas da extração e reconhecimento do \emph{audio fingerprint} possam ser modificadas sem que alterem o funcionamento geral do processo. Essa estrutura adaptável foi aplicada com base no modelo teórico genérico de reconhecimento de áudio.

% A terceira e última etapa foi o desenvolvimento de uma aplicação de reconhecimento específico de músicas, utilizando um cálculo de \emph{audio fingerprint} baseado em representações que levam em conta a repetição de refrões através da análise baseada em altura tonal e \emph{chroma} musical.

\section{Materiais Utilizados}
\label{s.materiaisutilizados}

\subsection{Ambiente de desenvolvimento}
\label{s.kali}

Para desenvolvimento do projeto foi utilizado o Sistema Operacional Kali Linux, uma distribuição Linux especializada em Testes de Intrusão e Auditoria de Segurança.

\subsection{Github}
\label{s.github}

O Github é ao mesmo tempo um servidor de armazenamento de código e uma rede social onde pode-se submeter modificações, fazer cópias e acompanhar modificações de códigos de outras pessoas. A rede foi essencial para o desenvolvimento deste projeto por armazenar vários sub-módulos e disponibilizar código-fonte para consulta.
