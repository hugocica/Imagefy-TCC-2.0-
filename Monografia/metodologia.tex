\chapter{Metodologia}
\label{c.metodologia}

\section{Métodos e Etapas}
\label{s.metodoseetapas}

Para o desenvolvimento do projeto foi realizado o levantamento de tecnologias atualmente existentes que podessem proporcionar o desenvolvimento da ferramenta. Por ter se planejado em ser uma aplicação Web, foi levado em consideração tecnologias para tal.

Com as tecnologias definidas, reuniu-se uma base teórica referentes à compressão de imagens, buscando ver novos métodos e quais seriam aplicados no desenvolvimento, bem como diferenciando diversos métodos já existentes com seus pontos positivos e negativos.

A última etapa do processo foi a de desenvolvimento da aplicação.

\section{Materiais Utilizados}
\label{s.materiaisutilizados}

\subsection{Ambiente de desenvolvimento}
\label{s.windows}

Para desenvolvimento do projeto foi utilizado o Sistema Operacional Windows 10, tendo em vista que o Servidor o qual a aplicação seria hospedada seria um ambiente Linux, distribuição Ubuntu.

\subsection{Github}
\label{s.github}

O Github é ao mesmo tempo um servidor de armazenamento de código e uma rede social onde pode-se submeter modificações, fazer cópias e acompanhar modificações de códigos de outras pessoas. A rede foi essencial para o desenvolvimento deste projeto por armazenar vários sub-módulos e disponibilizar código-fonte para consulta.

\subsection{TinyPNG}
\label{s.github}

TinyPNG é um aplicativo online semelhante à proposta desse projeto. Nele, é possível otimizar imagens de graça, com limite de 500 imagens/mês em uma conta gratuita. Esse aplicativo online foi utilizado para comparar o resultado da aplicação final.

\subsection{Codeigniter}
\label{s.ci}

Codeigniter é um framework para a linguagem PHP. Frameworks facilitam o desenvolvimento pois garantem uma aplicação estruturada, preparada para fácil manutenção e atualização. Além disso, são mais rápidos pois permite que códigos e bibliotecas sejam reaproveitadas, de modo que o programador foca apenas no desenvolvimento do módulo pretendido.
