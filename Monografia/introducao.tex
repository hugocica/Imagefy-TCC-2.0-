\chapter{Introdução}
\label{c.introducao}

Compressão de imagem permite reduzir seu tamanho em disco, buscando a menor perda de qualidade possível. A busca por isso se dá, nos dias de hoje, não somente pelo fator de utilização de espaço em disco, como também para facilitar acesso a navegação de sites e também sua rapidez. Quanto menor for o tamanho de um arquivo de imagem, menos dados serão requisitados, consumindo menores dados de tráfego. Esse último caso é um ponto importante, pois o crescimento de acesso a websites e aplicativos em dispositivos móveis tem crescido bastante nos útilmos anos. De acordo com o site Statista.com (https://www.statista.com/topics/779/mobile-internet/ - acessado em 27/05/2018), em 2018, o tráfico de internet acessado por celulares foi de 51.2\% em todo o globo. Tendo isso em mente, ao acessar um site contendo uma imagem em seu tamanho original, pode-se levar um tempo 5 vezes maior do que ao comprimi-la.

O tópico sobre compressão de imagens tem ganhado grande destaque por conta de perfomance, como também no quesito de armazenamento em disco. Para sanar a necessidade de falta de espaço em disco surgiram novas tecnologias, que é o caso dos Serviços em Nuvem. Esses Serviços permitem que sejam alocados recursos computacionais, cobrando um valor em cima do que se é realmente utilizado. Embora isso sane em parte o problema de armazenamento, ainda se é investido tecnologias e tempo para otimizar arquivos de imagens, pois reflete em um maior número de imagens em um servidor na nuvem pelo mesmo preço.

\section{Problema}
\label{s.problema}

Como comentado, a busca por esse tópico se dá por basicamente dois motivos: acessos mais rápidos à sites na Internet, como também menor utilização de espaço em disco.

Embora que nos dias atuais, HD's apresentam uma capacidade de armazenamento acima dos Terabytes, isso não é suficiente ainda assim. Embora que, de acordo com \citeauthoronline{linauthor} (\citeyear{linauthor}), 1 TB seja equivalente a 782.177 disquetes ou, 1.498 CD's, nos dias atuais, com a grande quantidade de consumo e armazenamento de dados, seria questão de dias até que o espaço fosse totalmente preenchido. A existência de Ambientes na Nuvem, que são sistemas remotos os quais se é possível alocar recursos computacionais a medida que se é necessário, ainda assim é desvantajoso não aproveitar totalmente o potencial de arquivos otimizados, pois se é cobrado pelo valor dos recursos utilizados.

Outro problema abordado é o fato de que arquivos de imagens consomem uma grande parte da banda, ao se carregar um site na Internet. Acessos por celulares nem sempre são feitos através de redes de internet WiFi, sendo muitas das vezes sendo acessados pela rua através de redes 3G/4G. Isso se torna um problema no consumo de dados, ao fazer requisição para arquivos muito grandes, além da demora para ser mostrada uma imagem, muitas vezes renderizadas em dimensões muito menor que a original.

Existem métodos, os quais iremos abortar, que tratam a otimização buscando sempre a menor perda de qualidade, ou dimensões da imagem. Seu uso irá variar com o que o usuário final busca.
