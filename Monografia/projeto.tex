\documentclass[
	% -- opções da classe memoir --
	12pt,				% tamanho da fonte
	openright,			% capítulos começam em pág ímpar (insere página vazia caso preciso)
	oneside,			% para impressão em verso e anverso. Oposto a oneside
	a4paper,			% tamanho do papel.
	% -- opções da classe abntex2 --
	%chapter=TITLE,		% títulos de capítulos convertidos em letras maiúsculas
	%section=TITLE,		% títulos de seções convertidos em letras maiúsculas
	%subsection=TITLE,	% títulos de subseções convertidos em letras maiúsculas
	%subsubsection=TITLE,% títulos de subsubseções convertidos em letras maiúsculas
	% -- opções do pacote babel --
	english,			% idioma adicional para hifenização
	french,				% idioma adicional para hifenização
	spanish,			% idioma adicional para hifenização
	brazil				% o último idioma é o principal do documento
	]{abntex2}

% ---
% Pacotes básicos
% ---
\usepackage{lmodern}			% Usa a fonte Latin Modern
\usepackage[T1]{fontenc}		    % Selecao de codigos de fonte.
\usepackage[utf8]{inputenc}		% Codificacao do documento (conversão automática dos acentos)
\usepackage{lastpage}			% Usado pela Ficha catalográfica
\usepackage{indentfirst}		    % Indenta o primeiro parágrafo de cada seção.
\usepackage{color}				% Controle das cores
\usepackage{graphicx}			% Inclusão de gráficos
\usepackage{microtype} 			% Para melhorias de justificação
\usepackage{afterpage}
\usepackage{amsmath}            % Pacote para fórmulas matemáticas
\usepackage{amssymb,url}
\usepackage{xcolor,tikz,bm,colortbl}
\usepackage[br]{nicealgo}       % Pacote para criação de algoritmos
\usepackage{customizacoes}

% ---

% ---
% Pacotes adicionais, usados apenas no âmbito do Modelo Canônico do abnteX2
% ---
\usepackage{lipsum}				% Para geração de dummy text
% ---

% ---
% Pacotes de citações
% ---
\usepackage[brazilian,hyperpageref]{backref}	 % Paginas com as citações na bibl
\usepackage[alf]{abntex2cite}	% Citações padrão ABNT
% ---
% CONFIGURAÇÕES DE PACOTES
% ---

% ---
% Configurações do pacote backref
\renewcommand{\familydefault}{\sfdefault}
% Usado sem a opção hyperpageref de backref
\renewcommand{\backrefpagesname}{Citado na(s) página(s):~}
% Texto padrão antes do número das páginas
\renewcommand{\backref}{}
% Define os textos da citação
\renewcommand*{\backrefalt}[4]{
	\ifcase #1 %
		Nenhuma citação no texto.%
	\or
		Citado na página #2.%
	\else
		Citado #1 vezes nas páginas #2.%
	\fi}%
% ---

% ---
% Informações de dados para CAPA e FOLHA DE ROSTO
% ---
\titulo{Métodos de Compressão de Imagem}
\autor{Hugo Cicarelli}
\local{Bauru}
\data{2018}
\orientador{Prof. Dra. Simone das Graças Domingues Prado}
\instituicao{%
  Universidade Estadual Paulista ``Júlio de Mesquita Filho''
  \par
  Faculdade de Ciências
  \par
  Ciência da Computação}
\tipotrabalho{Trabalho de Conclusão de Curso}
% O preambulo deve conter o tipo do trabalho, o objetivo,
% o nome da instituição e a área de concentração
\preambulo{Trabalho de Conclusão de Curso do Curso de Ciência da Computação da Universidade Estadual Paulista ``Júlio de Mesquita Filho'', Faculdade de Ciências, Campus Bauru.}
% ---


% ---
% Configurações de aparência do PDF final

% alterando o aspecto da cor azul
\definecolor{blue}{RGB}{41,5,195}

% informações do PDF
\makeatletter
\hypersetup{
     	%pagebackref=true,
		pdftitle={\@title},
		pdfauthor={\@author},
    	pdfsubject={\imprimirpreambulo},
	    pdfcreator={LaTeX with abnTeX2},
		pdfkeywords={abnt}{latex}{abntex}{abntex2}{trabalho acadêmico},
		colorlinks=true,       		% false: boxed links; true: colored links
    	linkcolor=black,          	% color of internal links
    	citecolor=black,        		% color of links to bibliography
    	filecolor=magenta,      		% color of file links
		urlcolor=black,
		bookmarksdepth=4
}
\makeatother
% ---

% ---
% Espaçamentos entre linhas e parágrafos
% ---

% O tamanho do parágrafo é dado por:
\setlength{\parindent}{1.3cm}

% Controle do espaçamento entre um parágrafo e outro:
\setlength{\parskip}{0.2cm}  % tente também \onelineskip

% ---
% compila o indice
% ---
\makeindex
% ---

% ----
% Início do documento
% ----
\begin{document}

% Seleciona o idioma do documento (conforme pacotes do babel)
%\selectlanguage{english}
\selectlanguage{brazil}

% Retira espaço extra obsoleto entre as frases.
\frenchspacing

% ----------------------------------------------------------
% ELEMENTOS PRÉ-TEXTUAIS
% ----------------------------------------------------------
% \pretextual

% ---
% Capa
% ---
\imprimircapa
% ---

% ---
% Folha de rosto
% (o * indica que haverá a ficha bibliográfica)
% ---
\imprimirfolhaderosto*
% ---

% ---
% Inserir a ficha bibliografica
% ---

% Isto é um exemplo de Ficha Catalográfica, ou ``Dados internacionais de
% catalogação-na-publicação''. Você pode utilizar este modelo como referência.
% Porém, provavelmente a biblioteca da sua universidade lhe fornecerá um PDF
% com a ficha catalográfica definitiva após a defesa do trabalho. Quando estiver
% com o documento, salve-o como PDF no diretório do seu projeto e substitua todo
% o conteúdo de implementação deste arquivo pelo comando abaixo:
%
% \begin{fichacatalografica}
%     \includepdf{fig_ficha_catalografica.pdf}
% \end{fichacatalografica}

\begin{fichacatalografica}
	\sffamily
	\vspace*{\fill}					% Posição vertical
	\begin{center}					% Minipage Centralizado
	\fbox{\begin{minipage}[c][8cm]{15.5cm}		% Largura
	\small
	\imprimirautor
	%Sobrenome, Nome do autor

	\hspace{0.5cm} \imprimirtitulo  / \imprimirautor. --
	\imprimirlocal, \imprimirdata-

	\hspace{0.5cm} \pageref{LastPage} p. : il. (algumas color.) ; 30 cm.\\

	\hspace{0.5cm} \imprimirorientadorRotulo~\imprimirorientador\\

	\hspace{0.5cm}
	\parbox[t]{\textwidth}{\imprimirtipotrabalho~--~\imprimirinstituicao,
	\imprimirdata.}\\

	\hspace{0.5cm}
		1. Image Compression
		2. Lossless
		3. Lossy
		4. Open Source
		I. \imprimirorientador.
		II. Universidade Estadual Paulista "Júlio de Mesquita Filho".
		III. Faculdade de Ciências.
		IV. Métodos Compressão de Imagem
	\end{minipage}}
	\end{center}
\end{fichacatalografica}
% ---

% ---
% Inserir folha de aprovação
% ---

% Isto é um exemplo de Folha de aprovação, elemento obrigatório da NBR
% 14724/2011 (seção 4.2.1.3). Você pode utilizar este modelo até a aprovação
% do trabalho. Após isso, substitua todo o conteúdo deste arquivo por uma
% imagem da página assinada pela banca com o comando abaixo:
%
% \includepdf{folhadeaprovacao_final.pdf}
%
\begin{folhadeaprovacao}

  \begin{center}
    {\ABNTEXchapterfont\large\imprimirautor}

    \vspace*{\fill}\vspace*{\fill}
    \begin{center}
      \ABNTEXchapterfont\bfseries\Large\imprimirtitulo
    \end{center}
    \vspace*{\fill}

    \hspace{.45\textwidth}
    \begin{minipage}{.5\textwidth}
        \imprimirpreambulo
    \end{minipage}%
    \vspace*{\fill}
   \end{center}

   \center Banca Examinadora

   \assinatura{\textbf{\imprimirorientador} \\ Orientador}
   \assinatura{\textbf{Profa Dra Andrea Carla Gonçalves Vianna} }
   \assinatura{\textbf{Prof Associado José Remo Ferreira Brega} }

   \begin{center}
    \vspace*{0.5cm}
    \par
    {Bauru}
    {2018}
    \vspace*{1cm}
  \end{center}

\end{folhadeaprovacao}
% ---

% ---
% Dedicatória
% ---
\begin{dedicatoria}
   \vspace*{\fill}
   \centering
   \noindent
   \textit{Agradeço a todos acontecimentos que me trouxeram aonde cheguei e me moldaram na pessoa que me tornei} \vspace*{\fill}
\end{dedicatoria}
% ---

% ---
% Agradecimentos
% ---
\begin{agradecimentos}
Quando penso na palavra agradecimentos, me aparecem vários rostos e nomes na cabeça. Isso de fato não é ao acaso, já que foram 8 anos de Unesp, 4 de TCC e muita torcida para esse momento.

Como é de costume, não poderia deixar de citar primeiro minha família, sem a qual eu não estaria aqui (literalmente). Meus pais sempre sonharam com esse momento e, claro, minha irmã a qual sempre foi meu modelo de vida, sempre me ajudando nas horas de aperto e me apertando nas horas que eu dava uma afroxada. Eternamente grato por esses seres.

Uma pessoa que devo muito também é alguém muito especial pra mim, que tornou o sentimento de lar bem real e a qual não fazia parte da minha vida há uns 2 anos. Provavelmente o maior ganho que Bauru me deu, que vou levar pra vida pois sei que sempre posso contar com ela e ela comigo. Seu nome é Mariana Saiani e com certeza grande parte desse projeto ter sido concluído se deve a ela, que sempre acreditou em mim mesmo quando eu desistia de mim mesmo. Não tenho palavras pra descrever o que ela significa pra mim, muito obrigado, nenezão.

Outros nomes que valem menção são dos meus dois irmão que encontrei nessa cidade. Cainã e Mário Sérgio, vou carregar vocês pra sempre em meu coração.

Válido também lembrar não só de alguém, mas eles formam a extensão da minha casa e merecem ser lembrados. Gaiola, vocês sabem o quanto significam pra mim!

Isabela Gouvêia é um nome razoavelmente novo, mas, como a própria mesma disse, nos conhecemos a tão pouco tempo e parece que já fazem anos. Obrigado pelas companhias durante horas de TCC, apenas existindo do meu lado. E, claro, por me alimentar durante esse período também.

Menina Priscila Beal é praticamente a alma desse projeto: fotógrafa por amor, a partir de conversas com ela que me veio aos olhos a urgência de mais aplicações como essa.

São tantos outros nomes igualmente especiais que não merecem passar batido. Bárbara, Laisla, vocês são mega especiais pra mim, obrigado por existirem!
\end{agradecimentos}
% ---

% \textemdash
% Epígrafe
% ---
\begin{epigrafe}
    \vspace*{\fill}
	\begin{flushright}
		\textit{Então é assim que a liberdade morre... com um estrondoso aplauso! - Star Wars}
	\end{flushright}
\end{epigrafe}
% ---

% ---
% RESUMOS
% ---

% resumo em português
\setlength{\absparsep}{18pt} % ajusta o espaçamento dos parágrafos do resumo
\begin{resumo}

Espaço destinado à escrita do resumo.

\textbf{Palavras-chave:} compressão de imagem, compactação, otimização, seo, jpeg, png.
\end{resumo}

% resumo em inglês
\begin{resumo}[Abstract]
 \begin{otherlanguage*}{english}

Abstract area.

\textbf{Keywords:} image compression, otimization, seo, lossy, lossless.

 \end{otherlanguage*}
\end{resumo}
% ---

% ---
% inserir lista de ilustrações
% ---
\pdfbookmark[0]{\listfigurename}{lof}
\listoffigures*
\cleardoublepage
% ---

% ---
% inserir lista de tabelas
% ---
\pdfbookmark[0]{\listtablename}{lot}
\listoftables*
\cleardoublepage
% ---

% ---
% inserir lista de abreviaturas e siglas
% ---
% ---

% ---
% inserir o sumario
% ---
\pdfbookmark[0]{\contentsname}{toc}
\tableofcontents*
\cleardoublepage
% ---



% ----------------------------------------------------------
% ELEMENTOS TEXTUAIS
% ----------------------------------------------------------
\pagestyle{simple}

% ----------------------------------------------------------
% Introdução (exemplo de capítulo sem numeração, mas presente no Sumário)
% ----------------------------------------------------------

\chapter{Introdução}
\label{c.introducao}

Compressão de imagem permite reduzir seu tamanho em disco, buscando a menor perda de qualidade possível. A busca por isso se dá, nos dias de hoje, não somente pelo fator de utilização de espaço em disco, como também para facilitar acesso a navegação de sites e também sua rapidez. Quanto menor for o tamanho de um arquivo de imagem, menos dados serão requisitados, consumindo menores dados de tráfego. Esse último caso é um ponto importante, pois o crescimento de acesso a websites e aplicativos em dispositivos móveis tem crescido bastante nos útilmos anos. De acordo com o site Statista.com (https://www.statista.com/topics/779/mobile-internet/ - acessado em 27/05/2018), em 2018, o tráfico de internet acessado por celulares foi de 51.2\% em todo o globo. Tendo isso em mente, ao acessar um site contendo uma imagem em seu tamanho original, pode-se levar um tempo 5 vezes maior do que ao comprimi-la.

O tópico sobre compressão de imagens tem ganhado grande destaque por conta de perfomance, como também no quesito de armazenamento em disco. Para sanar a necessidade de falta de espaço em disco surgiram novas tecnologias, que é o caso dos Serviços em Nuvem. Esses Serviços permitem que sejam alocados recursos computacionais, cobrando um valor em cima do que se é realmente utilizado. Embora isso sane em parte o problema de armazenamento, ainda se é investido tecnologias e tempo para otimizar arquivos de imagens, pois reflete em um maior número de imagens em um servidor na nuvem pelo mesmo preço.

\section{Problema}
\label{s.problema}

Como comentado, a busca por esse tópico se dá por basicamente dois motivos: acessos mais rápidos à sites na Internet, como também menor utilização de espaço em disco.

Embora que nos dias atuais, HD's apresentam uma capacidade de armazenamento acima dos Terabytes, isso não é suficiente ainda assim. Embora que, de acordo com \citeauthoronline{linauthor} (\citeyear{linauthor}), 1 TB seja equivalente a 782.177 disquetes ou, 1.498 CD's, nos dias atuais, com a grande quantidade de consumo e armazenamento de dados, seria questão de dias até que o espaço fosse totalmente preenchido. A existência de Ambientes na Nuvem, que são sistemas remotos os quais se é possível alocar recursos computacionais a medida que se é necessário, ainda assim é desvantajoso não aproveitar totalmente o potencial de arquivos otimizados, pois se é cobrado pelo valor dos recursos utilizados.

Outro problema abordado é o fato de que arquivos de imagens consomem uma grande parte da banda, ao se carregar um site na Internet. Acessos por celulares nem sempre são feitos através de redes de internet WiFi, sendo muitas das vezes sendo acessados pela rua através de redes 3G/4G. Isso se torna um problema no consumo de dados, ao fazer requisição para arquivos muito grandes, além da demora para ser mostrada uma imagem, muitas vezes renderizadas em dimensões muito menor que a original.

Existem métodos, os quais iremos abortar, que tratam a otimização buscando sempre a menor perda de qualidade, ou dimensões da imagem. Seu uso irá variar com o que o usuário final busca.

\chapter{Objetivos}
\label{c.objetivos}

\section{Objetivos Gerais}
\label{s.objetivosgerais}

Têm-se como objetivo final a criação de um aplicativo de Compressão de Imagens, o qual irá oferecer para o usuário o melhor cenário para o que se deseja, seja qualidade da imagem ou redimensionando-a.

\section{Objetivos Específicos}
\label{s.objetivosespecificos}

\begin{alineas}
	\item Aprender metodologias de compressão de imagem;
	\item Aprender algoritmos que permitem implementar as metodologias apresentadas;
	\item Melhorar habilidades com a linguagem PHP;
\end{alineas}

\chapter{Fundamentação Teórica}
\label{c.fundamentacaoteorica}

Com o contínuo aumento de dispositivos conectados à Internet, há também um grande aumento na ameaça de dados que podem estar comprometidos. Medidas de segurança são implementadas diariamente para impedir que usuários mal intencionados obtenham acesso à estas informações. Uma dessas medidas são os chamados Testes de Penetração, os quais são realizados sobre supervisão do dono do dispositivo, Sistema ou Rede de Computadores, e possuem o objetivo de descobrir possíveis vulnerabilidades, verificar quais informações se é possível obter, classificar estas vulnerabilidades e oferecer um relatório com suas descobertas e possíveis soluções para tais vulnerabilidades.

O intuito ao realizar Testes de Penetração, além de garantir que a segurança de um sistema esteja sempre preparada a novas ameaças, é o de verificar a aptidão a qual a equipe de uma empresa possui ao se deparar com problemas de invasão externa. Desta forma é possível classificar como Testes de Penetração como sendo um ramo da Segurança de Computadores.

\section{Segurança}
\label{s.seguranca}

Desde sempre, o valor da informação foi essencial, e nos tempos atuais não é diferente. Com o aumento da tecnologia e o grande avanço de dispositivos conectados à internet, existe uma grande preferência de manter informações virtualizadas, seja pela grande capacidade de armazenamento que os sistemas computacionais possuem, ou pelo fácil acesso, por exemplo. Porém, manter informações centralizadas em um só lugar pode se tornar um problema, caso não mantenha a segurança de seus dispositivos atualizada.

Para proteger dados e informações tendo em mente as diversas arquiteturas de rede que existem, {\em Web Services}, aplicações, diferentes plataformas de servidores, está mais difícil do que nunca. Por os computadores e a internet de fato estar presente no nosso dia-a-dia nas últimas décadas, invasões não são mais realizadas por crianças curiosas se aventurando no mundo dos códigos.

Apesar de existirem métodos os quais previnem o roubo de dados ou invasão de sistemas, como é o caso de Anti-vírus, por exemplo, o melhor jeito de descobrir se um sistema está realmente seguro contra invasões é tentando invadir o mesmo, pois apenas assim será possível detectar problemas reais os quais indíviduos mal intesionados podem vir a causar.

Estes usuários que se aproveitam de falhas do Sistema para ganho pessoal são comumente denominados como {\em hackers}.

De acordo com \citeauthoronline{hackerdictionary} (\citeyear{hackerdictionary}) define um {\em hacker} como sendo uma pessoa que contenha as seguinte características:

\begin{alineas}
  \item Uma pessoa a qual aprecia aprender detalhes de uma linguagem de programação ou de um sistema;
  \item Uma pessoa a qual aprecia programar ao invés de apenas teorizar;
  \item Uma pessoa capaz de apreciar o {\em hackeamento} de outra pessoa;
  \item Uma pessoa que aprende rapidamente uma linguagem de programação;
  \item Uma pessoa que é perito em determinada linguagem de programação ou sistema computacional.
\end{alineas}

Segundo \citeauthoronline{hackerdefinition}, pode-se separar {\em hackers} em dois grupos: {\em black hats} e {\em white hats}. {\em Black Hats} é uma vertente onde {\em hackers} utilizam seu conhecimento e habilidades para se aproveitar de falhas em Sistemas e Redes de Computadores, com o objetivo de incapacitar tais sistemas ou mesmo roubar informações críticas, utilizando-as para ganho pessoal. {\em White Hats}, muito semelhantes aos {\em black hats}, também invadem Sistemas e Redes de Computadores, porém com o intuito de buscar apenas descobrir vulnerabilidades, como é possível acessá-las, e buscar encontrar novas maneiras para reparar e prevenir estas e futuras falhas, ao invés de obter acesso e usufruir dos dados obtidos.

\section{Teste de Penetração}
\label{s.pentest}

No início deste capítulo, explicamos superficialmente que Testes de Penetração, ou também chamados Testes de Invasão ou {\em PenTest}, são testes realizados de modo a simular um ataque mal intencionado à uma rede, aplicação ou sistema computacional. Os testes consistem em:

\begin{alineas}
  \item Recolher informações do alvo, antes do teste (reconhecimento);
  \item Identificar possíveis pontos de entrada;
  \item Tentativa de invasão, seja virtualmente ou pessoalmente;
  \item Reportar as descobertas.
\end{alineas}

O objetivo de se realizar Testes de Invasão é determinar fraquezas na segurança justamente para reforçar a mesma nos sistemas testados, prevenindo assim que dados e informações sejam roubadas ou manipuladas por um {\em Cracker}. {\em Crackers}, como já foi dito, são {\em Hackers} que violam o Código de Ética dos {\em Hackers}, segundo \citeauthoronline{hackerdictionary}. A linha que separa um {\em PenTester} de um {\em Cracker} é exatamente pelo fato de um {\em PenTester} ter sido contratado para realizar o ato de invadir o sistema, tendo uma permissão prévia concedida pelo dono do sistema. Outro ponto que realizar testes de penetração podem contribuir, além de aumento da segurança, é testar a Política de Observação de uma empresa perante possíveis ataques, bem como testar o quão preparada está uma empresa para responder a possíveis ataques, e também verificar o quão ciente estão os funcionários quanto a brechas que podem resultar em uma invasão.

Testes de Invasão muitas vezes são confundidos com Avaliação de Vulnerabilidade. Porém isso é um erro comum, visto que o foco o qual se dá ao realizar um Teste de Penetração é na tentativa de ganhar um acesso ao sistema, a ponto de não existirem restrições a arquivos e dados sigilosos, enquanto que Avaliação de Vulnerabilidade tem ênfase em identificar pontos vulneráveis na segurança, não focando em tentar invadí-los.

Um {\em PenTester} tem como responsabilidade tentar invadir o sistema desejado de modo que irá usar os meios e métodos que um {\em Cracker} também usaria, para garantir que o sistema está seguro. Durante o processo de invasão, o {\em PenTester} estará fazendo um relatório detalhado, passando por todos os caminhos que ele tenha tomado, para que no final esteja especificado quais são as falhas do sistema, como foram descobertas e assim arrumá-las para que não haja potencial de invasão.

\subsection{Por que realizar Testes de Penetração?}
\label{ss.whypentest}

Precisa-se ter em mente de que uma falha corrigida ontem, pode resultar em uma falha imprevista hoje. Por conta disso, é importante manter constante as realizações de testes de penetração, garantindo assim um sistema seguro. Assim, é preciso estar sempre por dentro das novidades tecnológicas pois, atualizações podem ser facas de dois gumes: do mesmo modo que trazem novas medidas para prevenir quebras de segurança, podem também proporcionar novos meios para se invadir um sistema.

\citeauthoronline{pentestbefore} cita que o motivo principal para realização de Testes de Penetração é de encontrar vulnerabilidades através de ganho de acesso e resolver estas falhas. Porém, além desse motivo principal, também é citado que é uma boa prática ter o sistema verificado por olhos que não estavam inicialmente no projeto, podendo assim encontrar falhas não verificadas anteriormente. Outros motivos que são citados, podemos listar os seguintes:

\begin{alineas}
  \item Achar brechas antes que alguém mal intencionado ache: invasores podem estar explorando fraquezas a todo momento para tentar achar uma brecha. Como o autor cita, testes de penetração podem ser vistos como um Exame Médico Anual pois, não importa o quão saúdavel pareça, sempre é bom manter em dia os diagnósticos;
  \item Reportar problemas achados ao gerente responsável: muitas vezes um problema pode ser apontado, e o teste de penetração auxilia em maneiras de resolver o problema;
  \item Verificar configurações de segurança: realizar teste de invasão contribui para garantir que o sistema esteja realmente seguro contra invasões, além de que, segundo o autor, uma opinião externa ao projeto garante que todos os pontos de vistas não possuem falhas;
  \item Treinamento de segurança para a equipe: se um teste de penetração conseguir invadir com sucesso um sistema, isso pode indicar uma falta de treinamento por parte da equipe responsável pela segurança. Testes de penetração ajudam no treinamento da equipe, preparando para conseguirem identificar um ataque e impedí-lo antes que este seja bem sucedido;
  \item Testar novas tecnologias: de acordo com o autor, a melhor maneira para se testar tecnologias novas, é quando elas ainda não foram lançadas. Testes de penetração podem ajudar a revelar falhas desconhecidas antes de um lançamento, para que o produto seja totalmente confiável.
\end{alineas}

% Portanto, dá-se importância à realização de Testes de Penetração por conta de ... ?

\subsection{Estágios de um Teste de Penetração}
\label{ss.stagespentest}

\citeauthoronline{handson} (\citeyear{handson}) descreve os estágios que um Teste de Penetração percorre, que, de acordo com ela, totalizam em 7: Escopo, Reconhecimento, Modelo de Ameaça, Análise de Vulnerabilidade, Explorar Vulnerabilidades, Pós Exploração, Relatório.

A fase de Escopo é realizada antes mesmo de se começar o Teste. Consiste em um pré-engajamento, o qual envolve definir com o cliente os objetivos que o teste irá abordar. Nesta fase, o {\em PenTester} é encarregado de explicar ao cliente sobre como os testes ocorrerão, para que não haja falta de comunicação, o que pode resultar em uma intrusão além do esperado pelo lado do cliente. Este será o momento o qual se deve sentar com o cliente o que ele espera que um Teste de Penetração resulte, toda informação será crítica. Certas dúvidas terão que ser apresentadas, como o se terá que definir a maioria dos requisitos, quais portas de IP testar, quais ações serão permitidas pelo cliente, o trabalho será superficial, apenas verificando vulnerabilidades e brechas, ou poderá tentar derrubar o sistema, qual o horário que serão realizados os testes, entre outras perguntas técnicas. Por fim, é nesta fase que deverá ser providenciado um acordo, deixando claro que todas as informações obtidas serão mantidas confidencialmente, como também concedendo autorização para realizar os testes pois, caso contrário, poderá ser considerado um crime.

Após a fase de Escopo, ou Pré-engajamento, temos a fase de Reconhecimento. Durante essa fase, são analisadas informações sobre o alvo, bem como é feito uma análise de portas para se ter uma noção o tipo de sistema que se estará testando.

Modelo de Ameaça é a fase na qual se desenvolve como o ataque será realizado, a partir das informações reunidas na fase de Reconhecimento. Nesta fase será o momento no qual o {\em PenTester} pensa como um invasor e tentar explorar brechas que não seriam normalmente testadas, até conseguir algum acesso, caso estas existam.

Durante a fase de Análise de Vulnerabilidades, o {\em PenTester} fica responsável por descobrir vulnerabilidades que poderão vir a ser exploradas de modo que o teste seja bem sucedido. Nessa fase, será feita uma leitura das possíveis vulnerabilidades utilizando softwares que utilizam uma série de verificações. Porém não se pode confiar totalmente no software, sendo necessário também fazer uma análise manual, verificando a leitura dos resultados.

A quinta fase, a qual \citeauthoronline{handson} nomeou de Exploração de Vulnerabilidades, é onde o teste realmente ocorre. Nesta etapa, o {\em PenTester} irá se utilizar das brechas encontradas na etapa anterior e tentará explorar o máximo delas até que consiga realizar a invasão propriamente dita.

Na fase de de Pós Exploração, ocorre a análise dos dados coletados através da invasão, procurando por algum conteúdo com alto valor de informação, tentar conseguir acesso privilegiado ao sistema. De acordo com a autora, nesta parte será possível ver se as vulnerabilidades são realmente significativas, pois se somente for conseguido acesso a arquivos que não contém uma informação tão crítica, ainda será necessário resolvê-las, porém não são consideradas vulnerabilidades de alta prioridade.

A fase final será apresentado o relatório a respeito das descobertas encontradas. O relatório apresentado deve conter os pontos positivos a respeito do sistema do cliente, bem como deve estar bem detalhado a respeito de todos os passos tomados, como foi feita a invasão, para que seja possível reproduzir e, assim, resolver as vulnerabilidades encontradas. Um bom relatório pode ser dividido em Sumário Executivo e Relatório Técnico.
% \subsection{Teste de Penetração como Área Forense}
% \label{s.pentestforense}

\section{Computação na Nuvem}
\label{s.cloudcomputing}

Computação na Nuvem é um termo designado para especificar um serviço de hospedagem de serviços utilizando a Internet, ou seja, ao invés de ser necessário montar um servidor com todos os requisitos necessários, se é alugado um espaço de um servidor e seu acesso é feito pela Internet, pagando somente por aquilo que se é usado.

Dentre os diversos benefícios que a Computação na Nuvem trás, podemos listar:

\begin{alineas}
  \item Os recursos oferecidos suprem a maioria das demandas de quase todos os ramos de empresas;
  \item Elasticidade, ou seja, o poder de aumentar ou diminuir os recursos, conforme se é necessário;
  \item Pagar somente pelos recursos que são utilizados.
\end{alineas}

\subsection{Tipos de Serviços na Nuvem}
\label{s.cloudservices}

Segundo \citeauthoronline{clouddefinition}, em '{\em A comprehensive look at the path to cloud migrations}', Serviços de Computação na Nuvem são divididos nos seguintes tipos:

\begin{alineas}
  \item Privado;
  \item Público;
  \item Hibrido.
\end{alineas}

Serviço na Nuvem do tipo Privado ocorre quando o servidor, o qual estão todos os arquivos e dados, se localiza internamente na própria empresa, mesmo não estando no ambiente de trabalho das equipes. Normalmente existe uma sala para conter os diversos {\em hardwares} e recursos que suprem a necessidade. Esse tipo é bastante usado por empresas visto que, por estarem localizados internamente, são ambientes de trabalho mais seguros, por oferecer controle total dos recursos e não compartilhar estes recursos ou informações com outras empresas.

O segundo tipo abordado pela autora, Serviço de Nuvem Público, é o tipo oferecido pelos grandes {\em Datacenters} como Google ou Amazon, por exemplo, e pode ser contratado por qualquer empresa. É a opção mais acessível pois os recursos computacionais são oferecidos através da Internet, e o seu custo é somente por recursos utilizados pela empresa. Uma grande vantagem da Nuvem Pública é que é possível escalonar os recursos conforme for necessário, ou seja, se for necessários mais recursos computacionais, a empresa terá que configurar os recursos de forma a suprir as necessidades, podendo diminuir posteriormente, pagando apenas o que foi utilizado. A desvantagem que faz com que muitas empresas ficarem em dúvidas se migram para esse tipo de Computação na Nuvem é o fator de que os recursos oferecidos estão localizados externamente aos da empresa contratante, o que levanta desconfiança quanto à segurança de dados.

O último tipo citado, Nuvem Híbrida, é uma associação dos tipos privado e público. Com a nuvem híbrida, é possível tratar com informações sensíveis na nuvem privada, o que elimina incertezas e desconfianças relacionadas ao sigilo de informação. Ao mesmo tempo que arquivos menos críticos, como backup, email e armazenamento de dados estáticos, podem ser hospedados utilizando nuvem pública, o que proporciona maior elasticidade e menos custo em relação à recursos internos.

\subsection{Categorias de Serviços na Nuvem}
\label{s.cloudcategories}

\citeauthoronline{cloudexplained}, em seu artigo sobre introdução à Computação na Nuvem, cita os seguintes modelos de Serviços na Nuvem:

\begin{alineas}
  \item IaaS ({\em Infrastructure as a Service}): neste caso se tem acesso ao hardware de um sistema externo, como serviços de armazenamento ou servidores;
  \item PaaS ({\em Platform as a Service}): se é utilizado tanto o software quanto o hardware fornecidos pelo serviço contratado;
  \item SaaS ({\em Software as a Service}): significa que se utiliza o software do sistema oferecido.
\end{alineas}

% \subsection{Testes de Penetração em Sistemas na Nuvem}
% \label{s.pentestincloud}

\chapter{Metodologia}
\label{c.metodologia}

\section{Métodos e Etapas}
\label{s.metodoseetapas}

Para o desenvolvimento do projeto foi realizado o levantamento bibliográfico de metodologias existentes para realizar Testes de Penetração, analisando seus diversos aspectos e quais nos auxiliariam melhor em nosso objetivo, bem como modos de como aplicá-las em Sistemas na Nuvem..

Com a base teórica definida a etapa seguinte foi aplicar uma estrutura modular que correspondesse ao modelo teórico, para que todas as etapas da extração e reconhecimento do \emph{audio fingerprint} possam ser modificadas sem que alterem o funcionamento geral do processo. Essa estrutura adaptável foi aplicada com base no modelo teórico genérico de reconhecimento de áudio.

% A terceira e última etapa foi o desenvolvimento de uma aplicação de reconhecimento específico de músicas, utilizando um cálculo de \emph{audio fingerprint} baseado em representações que levam em conta a repetição de refrões através da análise baseada em altura tonal e \emph{chroma} musical.

\section{Materiais Utilizados}
\label{s.materiaisutilizados}

\subsection{Ambiente de desenvolvimento}
\label{s.kali}

Para desenvolvimento do projeto foi utilizado o Sistema Operacional Kali Linux, uma distribuição Linux especializada em Testes de Intrusão e Auditoria de Segurança.

\subsection{Github}
\label{s.github}

O Github é ao mesmo tempo um servidor de armazenamento de código e uma rede social onde pode-se submeter modificações, fazer cópias e acompanhar modificações de códigos de outras pessoas. A rede foi essencial para o desenvolvimento deste projeto por armazenar vários sub-módulos e disponibilizar código-fonte para consulta.

\chapter{Desenvolvimento}
\label{c.desenvolvimento}

\section{Utilizando Métodos {\em Lossless}}
\label{s.losslessdev}

Para o método sem perda escolhido foi a Codificação de Huffman, por ser uma técnica simples e fácil de aplicar. A técnica, como descrita no Capítulo 2, se baseia em remover a redundância de bits ao analisar diferentes características ou especificações.

\subsection{Codificação de Huffman}
\label{ss.huffmandev}

O primeiro passo nessa técnica é de reduzir a imagem original em um histograma ordenado, onde a probabilidade de ocorrência de um certo valor de intensidade de um pixel é igual a \[ PP = NP / T \] onde {\em PP} é a probabilidade de ocorrência, {\em NP} é o número de pixels de mesma intensidade e {\em T} o número total de pixels contidos na imagem original.

Histograma é a representação gráfica em colunas ou em barras de um conjunto de dados previamente tabulado e dividido em classes uniformes ou não uniformes.

Dada a seguinte imagem:

\begin{figure}[h]
\caption{\small Imagem 8x8 pixels}
\centering
\includegraphics[scale=0.50]{figs/Input-Image-1.png}
\label{f.imagecompressionbasics}
\end{figure}

Ao se fazer seu histograma, temos:

\begin{table}[]
    \begin{center}
        \caption{\small Valores de intensidade de pixels}
        \label{t.imagecompressionbasics}
        \begin{tabular}{llrlllll}
            128 & 75  & 72  & 105 & 149 & 169 & 127 & 100 \\
            122 & 84  & 83  & 84  & 146 & 138 & 142 & 139 \\
            118 & 98  & 89  & 94  & 136 & 96  & 143 & 188 \\
            122 & 106 & 79  & 115 & 148 & 102 & 127 & 167 \\
            127 & 115 & 106 & 94  & 115 & 124 & 103 & 155 \\
            125 & 115 & 130 & 140 & 170 & 174 & 115 & 136 \\
            127 & 110 & 122 & 163 & 175 & 140 & 119 & 87  \\
            146 & 114 & 127 & 140 & 131 & 142 & 153 & 93
        \end{tabular}
    \end{center}
\end{table}

A imagem contém 46 valores de intensidade distintos, portanto terá 46 símbolos únicos no dicionário de Huffman.

O método de Huffman pode ser separado da seguinte forma:

\begin{alineas}
    \item Construir a Árvore de Huffman
    \item Voltar todo o processo até que chegue ao nó inicial, atribuindo 0 ou 1 para cada nó intermediário, até que chegue ao nó da base.
\end{alineas}

\subsubsection{Construir a Árvore de Huffman}
\label{sss.huffmantree}

A figura \ref{f.imagebeezu} foi utilizada para testes durante o desenvolvimento.

\begin{figure}[h]
\caption{\small beezu.jpg, dimensão 1280x1280, 240KB}
\centering
\includegraphics[scale=0.25]{figs/beezu.jpg}
\label{f.imagebeezu}
\end{figure}

Para isso, precisamos primeiramente transformar a cadeia de caracteres da imagem em um vetor de caracteres. Tendo cada caracter num vetor, é feita uma contagem da frequência de cada caracter.

A partir disso é possível construir uma Árvore de Huffman. Isso é feito ao percorrer o {\em array} de frequências, inserindo cada caracter como um nó na Árvore de Huffman partindo pelos dois símbolos de menor frequência, que então são somados em símbolos auxiliares, sendo estes símbolos realocados no conjunto de símbolos. Tendo contruído a Árvore, percorremos ela, atribuindo valores binários para cada aresta.

\begin{figure}[h]
\caption{\small Representação da Árvore de Huffman da Figura 5}
\centering
\includegraphics[scale=0.08]{figs/beezu_huffman.PNG}
\label{f.imagebeezutree}
\end{figure}

\subsubsection{Reconstruir a Imagem Utilizando a Árvore de Huffman}
\label{sss.huffmantreebacktrack}

Começando do nó inicial, atribuimos a 0 para a esquerda e 1 para os nós da direita. Como estamos adicionando os nós recém formados ao {\em array} de frequência

\subsubsection{Resultados e Comparações}
\label{sss.results}
%
Pelos testes realizados, já foi possível notar uma diferença no tamanho da cadeia de caracteres gerada pela imagem. A imagem original, sua cadeia de caracteres possui um tamanho de 328864, enquanto que sua versão otimizada 197636.
% \begin{figure}[h]
% \caption{\small Fonte: Original - 240KB}
% \centering
% \includegraphics[scale=0.25]{figs/beezu.jpg}
% \label{f.imagebeezuo}
% \end{figure}
%
% \begin{figure}[h]
% \caption{\small Fonte: Testes - 213KB}
% \centering
% \includegraphics[scale=0.25]{figs/beezu.jpg}
% \label{f.imagebeezur}
% \end{figure}
%
% \begin{figure}[h]
% \caption{\small Fonte: TinyPNG - 141KB}
% \centering
% \includegraphics[scale=0.25]{figs/beezu.jpg}
% \label{f.imagebeezut}
% \end{figure}

\chapter{Conclusão}
\label{c.conclusao}

Tendo-se em mente os métodos estudados de compressão sem perda, o método de Huffman foi escolhido por ter sua aplicação simples e fácil, e um ponto importante foi a falta de pantente, tornando a utilização desse método em aplicações comerciais sem nenhum custo.

A partir do desenvolvimento e dos testes, foi possível ver que o método de Huffman é um método de fácil aplicação, porém ele por si só não é ótimo, visto que a mesma imagem utilizada otimizada pela ferramenta online TinyPNG, obteve melhor resultado na compressão. Como apontado durante a fase de pesquisa, é possível combinar esse método em questão com outros, buscando obter resultados melhores. Além de que, utilizando este método, o foco era testar metodologias {\em lossless}, as quais não sofrem perda de bits, podendo retornar para seu estado inicial.

Embora o algoritmo desse método seja ótimo para codificação símbolo a símbolo com uma distribuição de probabilidade conhecida, quando este caso não ocorre, não se torna tão vantajoso assim. Esse método pode ser utilizado em conjunto com outros, para garantir um melhor desempenho. Ferramentas de compactação, como o GZIP, se utilizam disso para realizar as compressões em sistemas Linux.

Durante o desenvolvimento, foram encontradas dificuldades durante o processo, principalmente para retornar ao estado de uma imagem, decodificando os resultados da Árvore de Huffman.

Atualmente, a busca por maior compressão de itens estáticos, como imagens, tem aumentado e muito. Com esse fator, o número de ferramentas disponíveis para tal tem aumentado também, porém a taxa de compressão é variada, sempre tendo uma busca pelo que melhor se adapa ao desejado. Ferramentas como TinyPNG, utilizada durante a fase de testes e comparação, ou Cloudinary, aplicação online a qual permite hospedagem de imagens e arquivos estáticos bem como sua otimização tanto na qualidade, como também usando métodos de recorte o qual também diminui seu espaço em disco, são amplamente usadas no dia a dia, tanto de aplicações mais gerais, como também de sites e blogs pela internet.



% ----------------------------------------------------------
% ELEMENTOS PÓS-TEXTUAIS
% ----------------------------------------------------------
\postextual
% ----------------------------------------------------------

% ----------------------------------------------------------
% Referências bibliográficas
% ----------------------------------------------------------
\pagestyle{empty}
\bibliography{references} % o arquivo de bibliografia deve ser importando nessa linha sem o .bib

% ----------------------------------------------------------
% Glossário
% ----------------------------------------------------------
%
% Consulte o manual da classe abntex2 para orientações sobre o glossário.
%
%\glossary

%---------------------------------------------------------------------
% INDICE REMISSIVO
%---------------------------------------------------------------------
\phantompart
\printindex
%---------------------------------------------------------------------

\end{document}
